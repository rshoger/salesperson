\documentclass{article}

\usepackage{algorithm}
\usepackage{algpseudocode}

\usepackage{enumitem}
\usepackage{amsmath, amssymb}

\usepackage[section]{placeins}

\setlength\parindent{0pt}

\begin{document}
\section{Nearest Neighbor Heuristic}

The nearest neighbor heuristic is one of the most straightforward approaches.
The heuristic is not so unlike how we would browse a museum. Similar to how we
would start by picking a point of interest, the nearest neighbor heuristic
begins at a single node. Then like touring a museum, we pick the next closest
exhibit or display, and continue going through the museum avoiding things we’ve
seen. The nearest neighbor heuristic works the same way, it identifies the
unvisited “nearest neighbor” to the current node, and traverses the graph until
all unvisited nodes have been seen once.

\subsection{Pseudocode}

\FloatBarrier

\begin{algorithm}
  \caption{Nearest\ Neighbor}
  \label{alg1}
  \begin{algorithmic}[1]
    \Procedure {Nearest\ Neighbor}{$\{G=(V,E),\ s\}$}

    \State $x \gets \textit{Random\ Value}$

    \State $L \gets \{\}$

    \State $s_{\textit{key} = 0}$
    \\
    \While {$|L| \neq |V|$}
      \For {\{ $u \in N(v) \mid v \in V$ \}}
        \State $x \gets \min N(v)$

        \If {$\textit{vertex } u \textit{not visited}$}
        \Else
          \State $x \gets \textit{Next shortest adjacency}$
        \EndIf
      \EndFor
    \EndWhile
    \\
    \State $L \gets L \cup {x}$
    \EndProcedure
  \end{algorithmic}
\end{algorithm}

\FloatBarrier

\subsection{Detail}
In additional to being fairly easily implemented, the nearest neighbor heuristic
is also a comparatively quick method. A major attraction of the nearest neighbor
is that it runs fairly quickly, it has a time complexity of $O(n^2)$
~\cite{nilsson2003heuristics}.\\

However there some rather large drawbacks to the approximation. A major flaw is
that in its greed, the tour can miss obvious shorter routes that could be
identified with more comprehensive algorithms. Additionally for the same reason,
the method can miss nodes till the end and need to include them at high cost.
In order to improve the path it would then require additional improvement
heuristics and approximation algorithms to decrease the cost of the tour and
increase quality. Though it was both simple and clear, for these reasons the
approach was not a suitable method for finding an optimal tour.

\end{document}
